 \documentclass[journal,12pt,twocolumn]{article}
\usepackage{blindtext}
\usepackage{graphicx}
\graphicspath{{./figs/}}
\usepackage{amsmath,amssymb,amsfonts,amsthm}
\usepackage{watermark}
\newcommand{\myvec}[1]{\ensuremath{\begin{pmatrix}#1\end{pmatrix}}}

\let\vec\mathbf
\thiswatermark{\centering \put(-20,-55){\includegraphics[scale=0.5]{iith.png}} }
\begin{document}
\title
{Matrix-Circle Assignment}
\author{Mohammad Imran}

\maketitle
\tableofcontents
\bigskip
\section{Problem Statement}
\ locus of the image of the point (2,3) in the line (2x-3y+4)+K
(x-2y+3)=0,K $\epsilon$ R,is a ?\\


\section{Construction}

\begin{figure}[h]
    \centering
\includegraphics[width=\columnwidth]{figure1.pdf}
    \caption{locus of the given point}
    \label{fig:my_label}
\end{figure}
\vspace{2cm}
\begin{table}[h]
    \centering
    \begin{tabular}{|c|c|c|}
       \hline
       \textbf{Symbol}&\textbf{Value}&\textbf{Description}  \\
       \hline
	    $\vec{P}$ & $\myvec{
		    x\\
		    y}$
	    & image of given point\\
        \hline
	    $\vec{Q}$ & $\myvec{1\\2}$
 & Point of Intersection\\
        \hline
	    $\vec{R}$ & $\myvec{
  2\\
  3}$
 & Given Point \\
        \hline
    \end{tabular}
    \caption{Parameters}
    \label{tab:my_label}
\end{table}

\section{Solution}
Given that resultant locus of the image of the point(2,3) in the line 2x-3y+4+K(x-2y+3)=0 here the K is a rational number assuming the locus of image of point (2,3) is P)\\


\textbf{Given family of the line:}
\vspace{0.3cm}
 \begin{center}
 (2x-3y+4)+K(x-2y+3)=0
 
 \end{center}
\vspace{0.3cm}
 
From the above given family of line we, can write as follows
\begin{equation}
\myvec{2&-3}\vec{X}=-4\label{eq-1}
\end{equation}
\begin{equation}
\myvec{1 & -2}\vec{X}=-3\label{eq-2}
\end{equation}

The above equation can be written in matrix form as follows
 
\begin{equation}
	\myvec{
  2  & -3\\
  1 & -2}\vec{X}=
  \myvec{
  -4 \\ -3}
  \label{eq-3}
\end{equation}
The Augmented can be expressed as 
\begin{equation}
  \myvec{
 2 & -3 &-4\\ 
 1 & -2 & -3}\label{eq-4}
\end{equation} 
Through,pivoting the Augmented matrix will become as 
\\
\begin{equation}
     \myvec{
 1 & 0\\ 
 0 & 1} \myvec{1 \\ 2} \label{eq-5}
 \end{equation}
On solving the above equation the cross point of the given family of line equation and we get point $\vec{Q}$
\begin{equation}
	\vec{Q}
  = \myvec{1\\
  2}   \label{eq-6}
\end{equation}
Given point $\vec{R}$=(2,3) and asumming its locus of image as point $\vec{P}$=(x,y) and we have obtain the point$\vec{Q}$=(1,2) \\


we, know that locus of the image of the point is equidistance from point Q 
thus,
\vspace{0.5cm}
\\ 
 \textbf{
Distance between point PQ and QR is same}\\
\begin{equation}
\vec{PQ}=\vec{QR}\label{eq-7}
\end{equation}


 \begin{equation}
 Distance between \vec{QR}=||\vec{Q}-\vec{R}||
\end{equation} 
 \begin{equation}
  \vec{QR}=||\myvec{1 
\\ 2}-\myvec{2 \\ 3}||
\end{equation} 
by, solving we get
\begin{equation}
 distance between \vec{QR}=\sqrt{2}
\end{equation}
 
From eq7 $\vec{PQ}=\sqrt{2}$

 \begin{equation}
 Distance between \vec{PQ}=||\vec{P}-\vec{Q}||
\end{equation} 
\begin{equation}
  ||\vec{P}-\vec{Q}||=\sqrt{2}
\end{equation} 
\begin{equation}
  ||\vec{P-Q}||^2=2
\end{equation} 

 by solving we get,
\begin{equation}
  {||\vec{P}||^2-2\vec{Q}^{\top}\vec{P}+||\vec{Q}||^2 
}=2
\end{equation} 

%the above equation resembels as,
%\begin{equation}
 % \myvec{a^{2}+b^{2}-2ga-2tb-c 
%}=0
%\end{equation} 
 %as the above equation is derived from equation of the circle,
%\begin{equation}
 % \myvec{
%{\vec{x^{\top}V x} + 2\vec{u^{\top}x}} + f}=0
%\end{equation} 
by solving we get equation of cirlce as,\\
 \begin{equation}
{{\vec{P^{\top}P} - 2\vec{Q^{\top}\vec{P}}+3}}=0 
 \end{equation}\\
 \begin{equation}
 {{\vec{X^{\top}X} - 2\myvec{-1 & -2}^{\top}\vec{X}}+3}=0
 \end{equation}
hence The Resultant is a \textbf{circle} \\

\section{finding the center of circle and radius}
 
 \begin{equation}
  center=-u=\myvec{1 \\ 2}
\end{equation} 

\begin{center}
radius of the circle=$\sqrt{2}$
\end{center}
\section{verfication of image of point with respect to line }
taken a line
\begin{equation}
 \vec{n}^{\top}\vec{X}=C
 \end{equation}\\
\begin{equation}
\myvec{1 & -2}\vec{X}=-3\label{eq-2}
\end{equation}

we have a point P =$\myvec{2 \\ 3}$

by using the formula image of point with respect to line
\begin{equation}
\vec{R}=\vec{P}+2\frac{C-\vec{n}^{\top}\vec{P}}{||\vec{n}||^2}\vec{n}
\end{equation}
by solving above equation we the image of point 
$\vec{R}=\myvec{\frac{12}{5}\\\frac{11}{5}}$
\section{Software}
Download the following code using,
\begin{table}[h]
    \centering
    \begin{tabular}{|c|}
    \hline \\
         https://github.com/imran111888/fwc2/blob//matrix/line/20assignment/codes/1 \\
         \\
\hline
    \end{tabular}
\end{table}
\\
and execute the code by using command
\begin{center}
\textbf{Python3  line.py}\\
\end{center}

\section{Conclusion}
We found the locus of the image of point(2,3) in a line is a circle.

\end{document}
