\documentclass[journal,12pt,twocolumn]{article}
\usepackage[english]{babel}
\usepackage[letterpaper,top=2cm,bottom=2cm,left=3cm,right=3cm,marginparwidth=1.75cm]{geometry}
\usepackage{multicol}
%\usepackage{amsmath}
\usepackage{graphicx}
%\usepackage{array}
\usepackage{blindtext}
\usepackage[utf8]{inputenc}
\usepackage{watermark}
\usepackage{hyperref}
\usepackage{fancybox}
%\usepackage[colorlinks=true, allcolors=blue]{hyperref}
\usepackage{listings}
\usepackage{float}
\usepackage{mdframed}

\title{FM}
\author{Under guidance of Dr.GVV SHARMA}
\thiswatermark{\centering \put(-50,-105){\includegraphics[scale=0.5]{iith.png}}}

\begin{document}
\maketitle
\tableofcontents
\section{Project Abstract}
Calculation of Bandwidth of Audio Signal
\section{Loading the Audio File}
We begin by loading the audio file using the wavfile.read() function from the scipy.io.wavfile module. This returns the sampling frequency and the audio data as a numpy array.
\section{Computing the Fourier Transform}
the Fourier transform of the audio signal using the numpy.fft.fft() function. This transforms the audio data from the time domain to the frequency domain.
\section{Calculating the Power Spectral Density}
The power spectral density (PSD) is a measure of the power of a signal as a function of frequency. We can calculate the PSD from the Fourier transform by taking the magnitude and squared magnitude of the transform.
\section{Finding the Frequency Range with Significant Power}
To calculate the bandwidth of the signal,  find the range of frequencies with significant power. We can do this by setting a threshold on the PSD and identifying the frequencies that exceed this threshold.set the threshold to 0.1 times the maximum PSD value.
\section{Calculating the Bandwidth}
Finally, we calculate the bandwidth as the difference between the maximum and minimum frequencies in the range with significant power.we obtain the bandwidth of the audio as 2 khz
\section{code link}
Python code to calculate the bandwidth of an audio signal.\\


\begin{mdframed}
https://github.com/imran111888/fwc2\\
/blob/main/FM/code/input.py

\end{mdframed}
just click on input.py \\

\fbox{\parbox{3cm}{\href{https://github.com/imran111888/fwc2/blob/main/FM/code/input.py}{input.py}}}



\end{document}